\section{External Conferences}

\subsection{Railway, Vehicular and Transportation}

\renewcommand{\tableentry}[5]{\multicolumn{3}{|l|}{\textbf{#1}}\\Date(s): #2&Location: #3&Interested Partners: #4\\\multicolumn{3}{|p{\textwidth}|}{#5}\\\hline}

\begin{longtable}{|lll|}
 \hline
 \tableentry{ERTMS World Conference}{?}{?}{?}{
	?
 }
 \tableentry{Signal und Draht}{?}{Fulda (D)}{?}{
	?
 }
 \tableentry{Moderne Schienenfahrzeuge}{?}{Graz (A)}{?}{
	TU Graz
 }
 \tableentry{Global Transport Forum Conference}{?}{?}{?}{
	Conference on Signalling and Train Control
 }
 \tableentry{Vehicular Conference}{?}{?}{?}{
	?
 }
 \tableentry{International Conference on Design and Operation in Railway Engineering (Comprail)}{?}{?}{?}{
	worldwide
 }
 \tableentry{International Railway Safety Conference (IRSC)}{?}{?}{?}{
	bi-annual, worldwide
 }
 \tableentry{Conference on Railway Engineering (CORE)}{?}{?}{?}{
	bi-annual, worldwide
 }
\end{longtable}

\subsection{Verification and Validation}


\begin{longtable}{|lll|}
 \hline
 \tableentry{European Safety and Reliability Conference (ESREL)}{?}{?}{?}{
	Safety, reliability and risk management become more and more important in an always more challenging and competitive environment, in every industry and human activity: multidisciplinary approaches to safety and reliability engineering and risk management become more and more necessary and attractive.

	This annual conference will provide a forum for presentation and discussion of scientific works covering theories and methods in the field of risk, safety and reliability, and their application to a wide range of industrial, civil and social sectors and problem areas. ESREL 2011 will also be an opportunity for researchers and practitioners, academics and engineers to meet, exchange ideas and gain insight from each other.
 }
\end{longtable}



\subsection{Software Engineering and Methodologies}

\begin{longtable}{|lll|}
 \hline
 \tableentry{Workshop on Certification and Model-Driven Development of Safety Critical Software (ZeMoSS)}{Feb. 2013}{Aachen (D)}{?}{
	The requirements for software quality of safety critical systems are very high, especially for the aspects of functional safety and information security. This German workshop covers among others the topics of model-driven development, integration of safety and security, and certification of open source software. Thus it is highly relevant for the project.
 }
\end{longtable}


\subsection{Open Source}

\subsection{General Computer Engineering and Computer Science}

\begin{longtable}{|lll|}
 \hline
 \tableentry{ETFA}{?}{?}{?}{
	?
 }
\end{longtable}
