\section{Communication}\label{sct:communication}
In the openETCS project we use a variety of different means of communication to enable the exchange of information between the project partners and external entities, to facilitate collaboration and to leverage the industrial, academic and societal awareness of the project, support the establishment of ecosystems of stakeholders and encourage the exchange of research results, thus contributing to the paradigm of open innovation.

\subsection{Internal Communication}

Internal communication enables the collaboration between the project partners. Internal communication takes place at different levels: global, work package-based and task-based. Thus, there is no information overflow and the project remains well-organised.

\subsubsection{Regular Teleconferences}

There are regular teleconferences for each work package which usually take place on a weekly basis in fixed time slots. The aim is to align the different partners with respect to the current goings-on in each work package. The organisation is Scrum-based, that is, each meeting is timeboxed and the participants are asked the following questions:
\begin{itemize}
\item What has been done in the last week?
\item What will be done next week?
\item Where there any impediments?
\end{itemize}

In addition, organisatorial issues are clarified and the alignment of the work package's tasks is ensured. For work packages with low activity, meetings take place at a lower frequency. The work package leaders organise the teleconferences and provide an agenda.

\subsubsection{Mailing Lists and Google Groups}

There are several mailing lists for the project, among them one for each work package, for global announcements, for the openETCS help desk, and for legal issues. While some (e.g., the project office mailing list) are restricted to certain members, most can be joined by any individual involved in the project. As they are hosted with Google Groups the whole communication is archived and can be accessed online.

\subsubsection{SharePoint}

The University of Rostock provides a SharePoint server that can be used to share documents within the project. The server is only accessible by project members and thus it is possible to provide restricted material for internal purposes.

\subsubsection{GitHub}

The public GitHub platform is used for management and publication of project results. It provides a wide variety of functions, including versioning, team collaboration support, public and private repositories and a wiki. It is especially useful if individuals are working on the same artifact simultaneously as it allows the tracking and merging of changes. The GitHub wiki is heavily used to provide organisatorial information as well as task-specific entries.

\subsection{External Communication}
\textcolor{red}{Could this be filled by DB?}

\subsubsection{openETCS Website}

