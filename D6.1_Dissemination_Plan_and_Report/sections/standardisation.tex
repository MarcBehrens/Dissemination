%%To be integrated in the exploitation plan, not in the dissemination plan
\section{Standardisation}\label{sct:standardisation}
\tbc[by DB]

Standardisation activities will ensure that the project results will feedback to the relevant standardisation bodies such as ERA, CENELEC, thus ensuring that the project results are aligned with recent advancements in the standardisation committees and that standardisation benefits from public funded knowledge and experience.

As the open source software and open proof philosophy is very innovative in railway control and protection applications, a significant impact from the openETCS project on the existing standards is expected. By providing impact analyses, the project results will be embedded in the CENELEC standards setting process
as well as in the development of the ETCS specifications by the ERA, in particular of the system requirements specification. Several partners of this ITEA2 Project (railway operators as well as manufacturers) are actively working in related standardisation committees.


\subsection{ETCS}

The system requirements specification (SRS) of ETCS has been published as an official document by the European Railway Agency (ERA), a governmental organization implemented by the European Commission. The purpose of the SRS is to specify the future unified standard European Train Control System ETCS from a technical point of view.


\subsection{CENELEC}

The European Standard EN 50128 from the ``European Committee for Electrotechnical Standardization''
(CENELEC) specifies procedures and technical requirements for the development of programmable electronic systems for use in railway control and protection applications. It is aimed for use in any area where there are safety implications.
