\section{Past Project Dissemination Activities}\label{sct:past_activities}

\begin{activity}{European Railway Review, Volume 18, Issue 3, 2012}
	\entry{Type}	{Journal Article}
	\entry{Title}	{openETCS: Applying `Open Proofs' to European Train Control}
	\entry{Pages} {30--34}
	\entry{Author}	{Klaus-R\"{u}diger Hase (DB)}
	\entry{Publisher} {Russell Publishing Ltd., Kent, UK}
\website{http://europeanrailwayreview.com/tag/openetcs}
	\desc{European Railway Review is the leading bi-monthly technical journal for the European rail industry. Featuring articles and news about the latest technologies and developments, the magazine is essential reading for people involved in the railway business. Here, an overview article of the openETCS project, covering the technical background and the project goals, has been published by DB Netz AG.}
\end{activity}

%==================================
% Missing input by ERTMS Solutions
%==================================
%\begin{activity}{FORMS/FORMAT 2012}
%	\entry{Type}	{Conference Paper and Talk}
%	\entry{Title}	{Using ERTMSFormalSpecs to model ERTMS braking curves}
%	\entry{Authors}	{L. Ferier, S. Lukicheva and S. Pinte (ERTMS Solutions)}
%	\entry{Date}		{11-13/12/2012}
%	\entry{Location}	{Braunschweig, Germany}
%	\website{http://www.forms-format.de}
%	\desc{\tbc[by ERTMS Solutions]}
%\end{activity}

%==================================
% Missing input by CEA LIST
%==================================
%\begin{activity}{10th International Conference on Software Engineering and Formal Methods (SEFM)}
%	\entry{Type}	{Conference Paper and Talk}
%	\entry{Title}	{Frama-C: a Software Analysis Perspective}
%	\entry{Author}	{Virgile Prevosto (CEA LIST)}
%	\entry{Date}		{01-05/10/2012}
%	\entry{Location}	{Thessaloniki (Greece)}
%	\entry{Publisher}	{Springer (LNCS)}
%	\website{http://sefm2012.city.academic.gr/}
%	\desc{\tbc[by CEA LIST], \red{is openETCS mentioned?}}
%\end{activity}

%==================================
% Missing input by CEA LIST
%==================================
%\begin{activity}{24th IFIP International Conference on Testing Software and Systems (ICTSS)}
%	\entry{Type}	{Conference Paper and Talk}
%	\entry{Title}	{Off-line test case generation for timed symbolic model-based conformance testing}
%	\entry{Author}	{Christophe Gaston (CEA LIST)}
%	\entry{Date}		{19-21/11/2012}
%	\entry{Location}	{Aalborg (Denmark)}
%	\entry{Publisher}	{Springer (LNCS)}
%	\website{http://ictss2012.aau.dk/}
%	\desc{\tbc[by CEA LIST], \red{is openETCS mentioned?}}
%\end{activity}


\begin{activity}{12th International SIGNAL+DRAHT-Congress 2012}
	\entry{Type}	{Talk}
	\entry{Title}	{openETCS: Von der Idee zur Praxis}
	\entry{Speaker}	{Klaus-R\"{u}diger Hase (DB)}
	\entry{Date}		{08-09/11/2012}
	\entry{Location}	{Fulda, Germany}
	\desc{The openETCS project has been presented at the congress, which took place under the motto ``How will signaling technology evolve over the next ten to fifteen years?''. The focus was predominantly on the development of the railway networks as well as on the expected changes in the areas of automatic
train control, route protection, level crossing protection and control technology. Therefore, this congress was an ideal platform to present the project to a large audience of highly distinguished experts.}
\end{activity}


\begin{activity}{APTA/UIC High-Speed Congress 2012}
	\entry{Type}	{Talk}
	\entry{Title}	{openETCS: Applying Open Proof's to the European Train Control System}
	\entry{Speaker}	{Klaus-R\"{u}diger Hase (DB)}
	\entry{Date}		{10-13/07/2012}
	\entry{Location}	{Philadelphia, PA, USA}
	\desc{The Congress focuses on highspeed railway traffic. Here, the openETCS project has been presented a large audience of highly distinguished experts and has especially been presented to the non-european audience.}
\end{activity}

%==================================
% Missing input by ERTMS Solutions
%==================================
%\begin{activity}{ERTMSFormalSpecs Workshop}
%	\entry{Type}{Workshop organised by ERTMS Solutions}
%	\entry{Date}{17-18/12/2012}
%	\entry{Location}{Brussels, Belgium}
%	\entry{Organisator}{ERTMS Solutions}
%	%\entry{Participants}{openETCS consortium, ERA, RFF, DVIS}
%	\desc{WP3a/b tasks contributions, \red{@ERTMS Solutions: Was this workshop open for external people to attend or just an internal event?}}
%\end{activity}

\begin{activity}{4th annual Signalling and Train and Control Conference}
	\entry{Type}	{Talk}
	\entry{Title}	{Infrastructure manager case study: Ensuring systems integration and developing functionality with openETCS}
	\entry{Speaker}	{Klaus-R\"{u}diger Hase (DB)}
	\entry{Date}		{19-21/03/2013}
	\entry{Location}	{Vienna, Austria}
	\entry{Website}		{\url{http://globaltransportforum.com/signalling-and-train-control/}}
	\desc{The 4th annual Signalling and Train Control show in Vienna is one of the definitive events for rail signalling, telecom and traffic management experts. With over 50 leading speakers and 300 attendees, the congress allows for the sharing of best practice strategies and unparalleled business development opportunities. At the symposium openETCS project leader Klaus-Rüdiger Hase presented the openETCS approach to developing functionality and system integration.}
\end{activity}



